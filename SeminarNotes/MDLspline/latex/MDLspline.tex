% Ԥ��Դ�ļ�

%% LyX 1.6.5 created this file.  For more info, see http://www.lyx.org/.
%% Do not edit unless you really know what you are doing.
\documentclass[english]{article}
\usepackage{lmodern}
\usepackage[T1]{fontenc}
\usepackage[latin9]{inputenc}
\usepackage{babel}
\usepackage{indentfirst}

\title{Splines, BLUPS, MDL and Knots Selection - ECNU Seminar Notes}
\author{Bu Zhou}
\date{2011,6,26}
\maketitle

\begin{document}
\section{Linear mixed models}
Linear mixed model is 

$y=X\beta+Zu+\epsilon$

where 

$E\left[\begin{array}{c}
u\\
\epsilon\end{array}\right]=\left[\begin{array}{c}
0\\
0\end{array}\right]$ and $Cov\left[\begin{array}{c}
u\\
\epsilon\end{array}\right]=\left[\begin{array}{cc}
G & 0\\
0 & R\end{array}\right]$

\section{Estimation for linear mixed models}
suppose:

$u\sim N(0,\sigma_{u}^{2}I_{m})$

$\epsilon\sim N(0,\sigma_{\epsilon}^{2}I_{n})$

$cov(u,\epsilon')=0$

$X_{n\times p}$ known matrix

$Z_{n\times m}$ known matrix

$u_{m\times1}$ vector of random effects

$y_{n\times1}$ data vector

$\beta_{p\times1}$ vector of unknown fixed effects parameters

we have:

$y\sim N(X\beta,\sigma^{2}V)$, $V=\lambda ZZ'+I_{n}$, $\lambda=\sigma_{u}^2/\sigma_{\epsilon}^2$
and $y|u\sim N(X\beta+Zu,\sigma_{\epsilon}^{2}I_{n})$ ({*}0)

penalized least squares criterion: $||y-X\beta-Zu||^{2}+\frac{1}{\lambda}||u||^{2}$({*}),
$\lambda>0$ is a tuning parameter.

For given $\lambda$, minimizing ({*}) with respect to $\beta$ and $u$ leads
to so-called mixed model equations

$\left(\begin{array}{cc}
X'X & X'Z\\
Z'X & Z'Z+\frac{1}{\lambda}I_{m}\end{array}\right)\left(\begin{array}{c}
\hat{\beta}\\
\hat{u}\end{array}\right)=\left(\begin{array}{c}
X'y\\
Z'y\end{array}\right)$

then:

$\hat{\delta}=\left(\begin{array}{c}
\hat{\beta}\\
\hat{u}\end{array}\right)=(M'M+\frac{1}{\lambda}D)^{-1}M'y$; $M=(XZ),$$D_{(p+m)\times(p+m)}=diag(0,\cdots,0,1,\cdots,1)$

the conditional ML(see({*}0)) estimator of $\sigma_{\epsilon}^{2}$
is $\hat{\sigma_{\epsilon}^{2}}=n^{-1}||y-M\hat{\delta}||^{2}=n^{-1}y'(I-H)^{2}y$

fitted values are $\hat{y}=Hy$; $H=M(M'M+\frac{1}{\lambda}D)^{-1}M'$
\newpage
\section{Penalized Splines as BLUPs}

(Brumback BA , Ruppert D , Wand MP (1999) Comment on Shively, Kohn
\& Wood . Journal of the American Statistical Association)

$y_{i}=f(x_{i})+\sigma_{\epsilon}\epsilon_{i},1\leq i\leq n,$

scatterplot data: $(x_{i},y_{i}),1\leq i\leq n$

$\epsilon_{i}$: independent $N(0,1)$ variates, $cov(\epsilon)=I$

$\kappa_{1},\cdots,\kappa_{K}$: set of distinct numbers inside the
range of the $x_{i}$'s 

$x_{+}=max(0,x)$

spline model: $f(x)=\beta_{0}+\beta_{1}x_{1}+\cdots+\beta_{p}x^{p}+\sigma_{u}\Sigma_{k=1}^{K}u_{k}(x_{i}-\kappa_{k})_{+}^p$

$u=\left[\begin{array}{c}
u_{1}\\
\vdots\\
u_{K}\end{array}\right]\sim N(0,I)$, is independent of $\epsilon=\left[\begin{array}{ccc}
\epsilon_{1} & \cdots & \epsilon_{n}\end{array}\right]^{T}$

$y=X\beta+\sigma_{u}Zu+\sigma_{\epsilon}\epsilon$, $\left[\begin{array}{c}
u\\
\epsilon\end{array}\right]\sim N(0,I)$

where

$X=\left[\begin{array}{cc}
1 & x_{1}\\
\vdots & \vdots\\
1 & x_{n}\end{array}\right]$ and $Z=\left[\begin{array}{ccc}
(x_{1}-\kappa_{1})_{+} & \cdots & (x_{1}-\kappa_{K})_{+}\\
\vdots & \ddots & \vdots\\
(x_{n}-\kappa_{1})_{+} & \cdots & (x_{n}-\kappa_{K})_{+}\end{array}\right]$

Estimation for given $\sigma_{u}$ and $\sigma_{\epsilon}$(replaced
by estimators $\hat{\sigma_{u}}$ and $\hat{\sigma_{\epsilon}}$),
the best linear unbiased predictor (BLUP) of $y$ is 

$\hat{f}=X\hat{\beta}+\sigma_{u}Z\hat{u}$

where $\hat{\beta}=\{X^{T}(\sigma_{u}^{2}ZZ^{T}+\sigma_{\epsilon}^{2}I)^{-1}X\}^{-1}X^{T}(\sigma_{u}^{2}ZZ^{T}+\sigma_{\epsilon}^{2}I)^{-1}y$
and $\hat{u}=(\sigma_{u}^{2}Z^{T}Z+\sigma_{\epsilon}^{2}I)^{-1}Z^{T}(y-X\hat{\beta})$

can be treated as an estimator of $f=[f(x_{1}),\cdots f(x_{n})]^{T}$(extends
to $f(x)$ for arbitrary $x$ is straightforward)

$\hat{f}$ can be rewritten as $\hat{f}=C(C^{T}C+\lambda D)^{-1}C^{T}y$

where $C=[XZ]$, $D=diag(0,0,1,1,\cdots,1)$ and $\lambda=\sigma_{\epsilon}^{2}/\sigma_{u}^{2}$.

\section{Minimum Description Length in the view of Bayesianism}

{[}Applying MDL To Learning Best Model Granularity, by Q.Gao, M.Li,
and P.M.B.Vitanyi, Artificial Intelligence, 2000{]}

\subsection{Introduction:}

Drawback of Person-Neyman testing:

Rejection of the zero hypothesis does not imply the acceptation of
alternative hypothesis. Does not establish the relative likelihood
between competing hypotheses.(All hypothesis different from the zero
hypothesis must be taken together to form the alternative hypothesis,
we can not even use the same data to test the alternative hypothesis
or a subset of it)

Bayesianism:

$P(H_{i}|D)=\frac{P(D|H_{i})P(H_{i})}{P(D)=\Sigma_{i}P(D|H_{i})P(H_{i})}$, 
select hypothesis/model with the maximum a posterior probability(MAP) 

where

$\Omega$: a discrete sample space

$D,H_{1},H_{2},\cdots$: a countable set of events(subsets) of $\Omega$

$H=\{H_{1},H_{2},\cdots\}$: hypothesis space

hypotheses $H_{i}$ are exhaustive(at least one is true), mutually
exclusive($H_{i}\cap H_{j}=\phi$ for all $i,j$)

Advantage: Allow to estimate the relative likelihood of different
possible hypotheses.

Disadvantage: Prior probability $P(H_{i})$, how to initially derive
it? May be unknown, uncomputable or conceivably nonexistent.

We know where to go next(Bayes's updating rule), but where shall we
start(prior)?

The answer: Find a single probability distribution to use as the prior
distribution in each different case, with approximately the same result
as if we had used the real distribution. 

Surprisingly, this solution turns out to be possible up to some mild
restrictions.

Universal prior in Bayes' rule: algorithmic universal probability
$m(x|y)=2^{-K(x|y)}$,where $K(x|y)$: prefix Kolmogrov complexity
of $x$ given $y$.

Problem: Cannot be directly used since Kolmogrov complexity $K(x|y)$
is non-computable, and so is the algorithmic universal probability
$m(x|y)$. Approximation is needed in the real world applications.

A {}``good'' computable approximation to $m(x)$ $\Rightarrow$MDL:

\subsection{MDL in one page:}

From Bayes' formula, we must choose the hypothesis $H$ that maximizes
the posterior $P(H|D)$, taking the negative logarithm on both side,

$-logP(H|D)=-logP(D|H)-logP(H)+logP(D)$

$logP(D)$ is a constant and can be ignored because we just want to
optimize the left-hand side of the equation over $H$.

The problem is minimizing $-logP(D|H)-logP(H)$ = 

- (the log universal probability of the model + the log of the probability
of the data given the model)

Ideal MDL: $K(H)+K(D|H)$ (use universal prior)

Real MDL: $-logP(D|H)-logP(H)$, $P(D|H)$ must be computable. (Applied
statistical version of MDL, use Shannon-Frano code as the approximation
of the non-computable Kolmogrov complexity)

The Shannon-Frano code assigns code words of $length\stackrel{+}{=}-logP(.)$
to elements randomly drawn according to a probability $P(.)$.

$-logP(H)$: the length, in bits, of the description of the theory,
$=K(H)$, provided that $-logP(H)\stackrel{+}{=}-logm(H)$

$-logP(D|H)$: the length, in bits, of data when encoded with the
help of the theory, $=K(D|H)$ provided that $-logP(D|H)\stackrel{+}{=}-logm(D|H)$

For {}``typical'' outcomes, $K(D|H)\stackrel{+}{=}-logP(D|H)$ means
that the classic Shannon-Frano code length reaches the prefix Kolmogorov
complexity on these data samples. (Under the assumption that the data
sample is typical for the contemplated hypotheses, the ideal MDL principal
and the applied statistical one coincide, and moreover, both are valid
for a set of data samples of Lebesgue measure one.)
\subsection{Examples}
\textbf{Example 1.} (Hypothesis Testing) 

$H$: some model $H(\theta)$ with a set of parameter $\theta=\{\theta_{1},\cdots,\theta_{k}\}$
of precision $c$.

number $k$ may vary and influence the description complexity of $H(\theta)$.

For example, if we want to determine the distribution of the length
of beans, then $H$ is a normal distribution $N(\mu,\sigma)$ with
parameter median $\mu$ and variation $\sigma$. So essentially we
have to determine the correct hypothesis described by identifying
the type of distribution(normal) and the correct parameter vector
$(\mu,\sigma)$. 

In such cased, we minimize $-logP(D|\theta)-logP(\theta)$.

\textbf{Example 2.} (Fitting of a 'best' polynomial on $n$ given sample points
in the 2-dimensional plane.)

For each fixed $k$, $k=0,\cdots,n-1$, let $f_{k}$ be the best polynomial
of degree $k$, fitting on points $(x_{i},y_{i})(1\leq i\leq n)$,
which minimizes the error

$error(f_{k})=\Sigma_{i=1}^{n}(f_{k}(x_{i})-y_{i})^{2}$.

Assume each coefficient takes $c$ bits, so $f_{k}$ is encoded in
$c(k+1)$ bits.

Assume $Pr(y_{1},\cdots,y_{n}|f,x_{1},\cdots,x_{n})=\prod exp\{-O[(f(x_{i})-y_{i})^{2}]\}$.(measurement
with Gaussian/normal errors), so $-log(Pr(y_{1},\cdots,y_{n}|f,x_{1},\cdots,x_{n}))=c'error(f)$
for some computable $c'$.

The MDL principle tells us to choose $f=f_{m}$ with $m\in\{0,\cdots,n-1\}$,
which minimizes $c(m+1)+c'error(f_{m})$.

\section{Model selection by the MDL principle}

data: $\{x_{i},y_{i}\}_{i=1}^{n},y_{i}=f(x_{i})+\epsilon_{i},\epsilon_{i}\sim iidN(0,\sigma^{2})$

$f$ can be well approximated by $r$-order regression spline with
$m$ knots: $f(x)\approx b_{0}+b_{1}x+\cdots+b_{r}x^{r}+\Sigma_{j=1}^{m}\beta_{j}(x-k_{j})_{+}^{r}$

$k_{j}$: location of the $j$th knot

$\{b_{0},\cdots,b_{r},\beta_{1},\cdots,\beta_{m}\}$: set of the coefficients

$(a)_{+}=max(0,a)$

$(a)_{+}^{c}=((a)_{+})^{c}$

assume $min(x_{i})<k_{1}<\cdots<k_{m}<max(x_{i})$ and $\{k_{1},\cdots,k_{m}\}$
is subset of $\{x_{1},\cdots,x_{m}\}$

estimation of $f$ can be transformed into a model selection problem
with each plausible model $\theta$ completely specified by $\theta=\{r,m,k,b,\beta\}$

note that different $\theta$'s may have different dimensions (number
of parameters)

Let

$x=(x_{1},\cdots,x_{n})^{T}$, $y=(y_{1},\cdots,y_{n})^{T}$ and $M=(XZ)=(1,x,\cdots,x^{\hat{r}},(x-\hat{k_{1}}1_{n\times1})_{+}^{\hat{r}},\cdots,(x-\hat{k}_{\hat{m}}1)_{+}^{\hat{r}})$


\subsection{regression spline}

the natural estimates(maximum likelihood estimates conditional on
$\hat{r}$, $\hat{m}$ and $\hat{k}$) of $b$ and $\beta$ are given
by 

$(\hat{b}^{T},\hat{\beta}^{T})^{T}=(M^{T}M)^{-1}M^{T}y$ 

we estimate $\theta$ and hence $f$ by minimizing the 

$MDL(\hat{\theta})=
\begin{array}{c}
log\hat{m}+\Sigma_{j=1}^{\hat{m}}log\hat{l_{j}}+\frac{1}{2}\Sigma_{j=1}^{\hat{m}+1}log\hat{l_{j}}+\frac{n}{2}log\{\frac{RSS(\hat{\theta})}{n}\}, \hat{r}=0;\\
log\hat{m}+\Sigma_{j=1}^{\hat{m}}log\hat{l_{j}}+\frac{\hat{m}+\hat{r}+1}{2}logn+\frac{n}{2}log\{\frac{RSS(\hat{\theta})}{n}\}, hat{r}\geq1.\end{array}$ 

\textbf{Example 3.} (Derivation of code length expression for choosing knots
in regression spline smoothing)

$L(y)=L$(fitted model)$+L$(data given the fitted model)$=L(\hat{\theta})+L(y|\hat{\theta})$.

1. $L(\hat{\theta})=L(\hat{r})+L(\hat{m})+L(\hat{k}|\hat{m})+L(\hat{b},\hat{\beta}|\hat{r},\hat{m},\hat{k})$

code length for $\hat{r}$ and $\hat{m}$($L(\hat{r})+L(\hat{m})$): 

use $r\in\{0,1,2,3\}$ $\Longrightarrow$$L(\hat{r})=-log_{2}\frac{1}{4}=log_{2}4=2$
bits, a constant that will be ignored.

$\hat{m}$ is an integer, $L(\hat{m})=L^{*}(\hat{m})=log_{2}c+log_{2}\hat{m}+log_{2}log_{2}\hat{m}+\cdots$,
where $c\approx2.865$, $L^{*}(\hat{m})\approx log_{2}\hat{m}$when
$\hat{m}$ is reasonably large.

Thus $L(\hat{r})+L(\hat{m})\approx log_{2}\hat{m}$.

code length for $\hat{k}$ given $\hat{m}$($L(\hat{k}|\hat{m})$): 

note that $\{\hat{k_{1}},\dots,\hat{k_{\hat{m}}}\}$ is restricted
to be a subset of $\{x_{1},\cdots,x_{n}\}$.

$L(\hat{k}|\hat{m})=L(\hat{l_{1}},\cdots,\hat{l_{\hat{m}}}|\hat{m})=\Sigma_{j=1}^{\hat{m}}L^{*}(\hat{l_{j}})\approx\Sigma_{j=1}^{\hat{m}}log_{2}\hat{l_{j}}$.

where $\hat{l_{j}}$(integers) is the $j$th successive {}``index
difference'', the number of $x_{i}$'s which satisfy $\hat{k_{j-1}}\leq x_{i}<\hat{k_{j}},j=1,\cdots,\hat{m};\hat{k_{0}}=min(x_{i}),\hat{k_{\hat{m}+1}}=max(x_{i})$.

(complete knowledge of $\hat{l_{1}},\cdots,\hat{l_{\hat{m}}}$ implies
complete knowledge of $\hat{k}$.)

code length for $\{\hat{b},\hat{\beta}\}$ given $\{\hat{r},\hat{m},\hat{k}\}$($L(\hat{b},\hat{\beta}|\hat{r},\hat{m},\hat{k})$):

each of $\{\hat{b},\hat{\beta}\}$- $\hat{b}_{i}$ or $\hat{\beta}_{i}$,
is (conditional) maximum likelihood estimate from $n$ data points
when $\hat{r}\geq1$, so it can be effectively encoded with $\frac{1}{2}log_{2}n$
bits. 

$L(\hat{b_{0}})=\cdots=L(\hat{b_{\hat{r}}})=L(\hat{\beta_{1}})=\cdots=L(\hat{\beta_{\hat{m}}})=\frac{1}{2}log_{2}n$,
so

$L(\hat{b},\hat{\beta}|\hat{r},\hat{m},\hat{k})=\frac{\hat{r}+\hat{m}+1}{2}log_{2}n$,
when $\hat{r}\geq1$.

$L(\hat{b},\hat{\beta}|\hat{r},\hat{m},\hat{k})=\Sigma_{j=1}^{\hat{m}}L$({}``$j$th
estimated height'')$=\frac{1}{2}\Sigma_{j=1}^{\hat{m}+1}log_{2}\hat{l_{j}}$,
when $\hat{r}=0$.

code length for $y$ given $\hat{\theta}=\{\hat{r},\hat{m},\hat{k},\hat{b},\hat{\beta}\}$($L(y|\hat{\theta})$):

$L(y|\hat{\theta})=\frac{n}{2}log_{2}\{\frac{RSS(\hat{\theta})}{n}\}+C$,
where $C$ is a negligible term and
$RSS(\hat{\theta})=\Sigma\{y_{i}-\hat{f}(x_{i})\}^{2}$. 


\subsection{smoothing spline/penalized spline}

$(\hat{b}^{T},\hat{\beta}^{T})^{T}=(M^{T}M+\frac{1}{\lambda}D)^{-1}M^{T}y$

consider a set of normal models of form $y|b_{\eta}\sim N(X_{\eta}\beta_{\eta}+Z_{\eta}b_{\eta},\sigma^{2}I_{n})$

$\eta$:index of the set of candidate models

$X_{\eta}$: $n\times r_{\eta}$ matrix

$Z_{\eta}$: $n\times m_{\eta}$ matrix

$\beta_{\eta}$: $r_{\eta}\times1$ parameter vector

$b_{\eta}$: $m_{\eta}\times1$ parameter vector

Note that estimates $\beta_{\eta}$, $b_{\eta}$ and $\hat{\sigma}_{\eta}^{2}$
depends on the parameter $\lambda\in[0,\infty]$

we specify a model by giving the pair $\gamma=(\eta,\lambda)$

assume response data are modelled with a set of density functions
$f(y;\gamma,\theta)$, where parameter vector $\theta$ varies within
a specified parameter space,

the NML(normalized maximum likelihood) function is defined by

$\hat{f}(y;\gamma)=\frac{f(y;\gamma,\hat{\theta})}{C(\gamma)}$,

where 

$\hat{\theta}=\hat{\theta}(y)$ is the ML estimator of $\theta$ and 

$C(\gamma)=\int f(x;\gamma,\hat{\theta}(x))dx$ is the normalizing
constant.

integral is taken over the sample space

$\hat{f}(y;\gamma)$ defines a density function provided that $C(\gamma)$
is bounded.

shortest code length: $-log\hat{f}(y;\gamma)=-logf(y;\gamma,\hat{\theta})+logC(\gamma)$ 

$f(y;\gamma,\hat{\theta})=(2\pi)^{-\frac{n}{2}}|\hat{\Sigma}|^{-\frac{1}{2}}exp\{-\frac{1}{2}(y-(X\hat{\beta}+Z\hat{b}))^{T}\hat{\Sigma}{}^{-1}(y-(X\hat{\beta}+Z\hat{b}))\}$

where $\hat{\Sigma}=\hat{\sigma^{2}}I$,

$\hat{\sigma^{2}}=n^{-1}||y-\hat{y}||^{2}=n^{-1}||y-(X\hat{\beta}+Z\hat{b})||^{2}=n^{-1}||y-M^{T}\hat{\theta}||^{2}=n^{-1}y^{T}||I-H||^{2}y$,
$H=(M^{T}M+\frac{1}{\lambda}D)^{-1}M^{T}$.

then we have $f(y;\gamma,\hat{\theta})=(2\pi)^{-\frac{n}{2}}|n^{-1}||y-\hat{y}||^{2}I_{n}|^{-\frac{1}{2}}exp\{-\frac{1}{2}trace(I_{n})\}=(2\pi)^{-\frac{n}{2}}\{n^{-1}||y-\hat{y}||^{2}\}^{-\frac{n}{2}}exp\{-\frac{1}{2}n\}$

$-log_{2}f(y;\gamma,\hat{\theta})=\frac{n}{2}log_{2}(2\pi)+\frac{n}{2}log_{2}\{n^{-1}||y-\hat{y}||^{2}\}+log_{2}\{exp(-\frac{1}{2}n)\}=\frac{n}{2}log_{2}\{\frac{RSS(\hat{\theta})}{n}\}+C$,

where $RSS(\hat{\theta})=n^{-1}||y-\hat{y}||^{2}=\Sigma_{i=1}^{n}(y_{i}-\hat{y_{i}})^{2}$
is the residual sum of squares.

$C=\frac{n}{2}log_{2}(2\pi)+log_{2}\{exp(-\frac{1}{2}n)\}$ is a negligible
term.

\subsection{Knots deletion algorithm}

Problem: finding the global minimizer of $MDL(\hat{\theta})$ is difficult(Due
to the complexity of $\hat{\theta}$)- a global search is infeasible
even if $n$ is only of moderate size.

Disadvantage: miss the global minimizer 

Advantage: guarantees to find a local minimizer

Algorithm:

1. fixed a value for $\hat{r}$(spline order, 0,1,2,3), placing(every
$s$(3\textasciitilde{}5) sorted values of $x_{i}$'s/equally spaced
sample quantiles of $x_{1}\cdots x_{n}$) a relatively large number($K$)
of initial knots, (overfitted model) compute $MDL(\hat{\theta})$

2. for each knot in all knots, remove this knot, recompute $MDL(\hat{\theta})$;
choose/delete/record the knot s.t. $MDL(\hat{\theta})$1-$MDL(\hat{\theta})$2=max
(greedy strategy)

3. goto 1 until all initial knots are removed

4. choose the one has the smallest $MDL(\hat{\theta})$ value with
respect to $K$ ($K+1$models in total) as the best fitted model for
that fixed value of $\hat{r}$

5. change the value of $\hat{r}$, goto 1 until all candidate values(0,1,2,3)
are computed

(5+.change the value of $\lambda$, goto 1 until all candidate values
are computed)

6. chose the model has smallest $MDL(\hat{\theta})$ with respect
to $\hat{r}$(and $\lambda$) as the final model

\section{References:}

\subsection{Books:}

{[}Semiparametric Regression by David Ruppert, M. P. Wand, R. J. Carroll,
2003{]}

{[}Longitudinal Data Analysis, Edited by Garrett Fitzmaurice, Marie
Davidian, Geert Verbeke \& Geert Molenberghs, 2008{]}

{[}Stochastic Complexity in Statistical Inquiry, by Jorma Rissanen,
World Scientific, 1989{]}

{[}Information and Complexity in Statistical Modeling, by Jorma Rissanen,
Springer, 2007{]}

{[}The Minimum Description Length Principle, by Peter Grunwald, MIT
Press, 2007{]}

\subsection{Papers:}

{[}Applying MDL To Learning Best Model Granularity, by Q.Gao, M.Li,
and P.M.B.Vitanyi, Artificial Intelligence, 2000{]}

{[}Comment on Shively, Kohn and Wood, by Babette A. Brumback , David
Ruppert , M. P. Wand, Journal of the American Statistical Association,
1999{]}

{[}Regression Spline Smoothing Using The Minimum Description Length
Principle, by Thomas C. M. Lee, Statistics \& Probability Letters,
2000{]}

{[}MDL Knot Selection For Penalized Splines, by Antti Liski and Erkki
P. Liski, WITMSE, 2008{]}

{[}Model Selection In Linear Mixed Models Using MDL Criterion With
An Application To Spline Smoothing, by Erkki P. Liski and Antti Liski,
WITMSE, 2008{]}

{[}http://www.mdl-research.org/{]}
\end{document}
