%% LyX 1.6.5 created this file.  For more info, see http://www.lyx.org/.
%% Do not edit unless you really know what you are doing.
\documentclass[english]{article}
\usepackage{lmodern}
\usepackage[T1]{fontenc}
\usepackage[latin9]{inputenc}
\usepackage{textcomp}
\usepackage{amstext}
\usepackage{esint}
\usepackage{longtable}
\usepackage{rotating}
\usepackage{indentfirst}
\makeatletter

%%%%%%%%%%%%%%%%%%%%%%%%%%%%%% LyX specific LaTeX commands.
\newcommand{\tabincell}[2]{\begin{tabular}{@{}#1@{}}#2\end{tabular}}
\providecommand*{\perispomeni}{\char126}
\AtBeginDocument{\DeclareRobustCommand{\greektext}{%
  \fontencoding{LGR}\selectfont\def\encodingdefault{LGR}%
  \renewcommand{\~}{\perispomeni}%
}}
\DeclareRobustCommand{\textgreek}[1]{\leavevmode{\greektext #1}}
\DeclareFontEncoding{LGR}{}{}

\newcommand{\lyxmathsym}[1]{\ifmmode\begingroup\def\b@ld{bold}
  \text{\ifx\math@version\b@ld\bfseries\fi#1}\endgroup\else#1\fi}

%% Because html converters don't know tabularnewline
\providecommand{\tabularnewline}{\\}

\makeatother

\usepackage{babel}
\title{Spline Basics - ECNU Seminar Notes}
\author{Bu Zhou}
\date{2011,6,21}
\maketitle
\begin{document}
\section{Three different kinds of splines}
$y=u(t)+\epsilon$,

$\epsilon\sim(N)(0,\sigma^{2}I_{n})$;

$y_{i}=u(t_{i})+\epsilon_{i}$,

$\epsilon_{i}\sim(N)(0,\sigma^{2})$;

\subsection{Interpolating<->Smoothing:} 
\subsection{Different Splines}

Linear Regression-> Polynomial Regression-> Orthogonal Polynomial Regression-> Piecewise Polynomial Regression


\begin{tabular}{|c|c|}
\hline 
Linear Regression & Polynomial Regression\tabularnewline
\hline 
$Y=X\beta+\epsilon$;$y=\left(\begin{array}{c}
y_{1}\\
\vdots\\
y_{n}\end{array}\right)$,$\beta=\left(\begin{array}{c}
\beta_{0}\\
\vdots\\
\beta_{t}\end{array}\right)$ & \tabularnewline
\hline 
$X_{L}=\left(\begin{array}{cccc}
1 & X_{11} & \cdots & X_{1t}\\
\vdots & \vdots & \ddots & \vdots\\
1 & X_{n1} & \cdots & X_{nt}\end{array}\right)$ & $X_{P}=\left(\begin{array}{ccccc}
1 & x_{1} & x_{1}^{2} & \cdots & x_{1}^{t}\\
\vdots & \vdots & \vdots & \ddots & \vdots\\
1 & x_{n} & x_{n}^{2} & \cdots & x_{n}^{t}\end{array}\right)$\tabularnewline
\hline 
 & \tabularnewline
\hline 
$M(\beta)=RSS(\beta)=\Sigma_{i=1}^{n}(y_{i}-x_{i}\beta)^{2}$ & $M(\beta)=RSS(\beta)$\tabularnewline
\hline
\end{tabular}

\begin{tabular}{|c|}
\hline 
Orthogonal Polynomial Regression\tabularnewline
\hline 
\tabularnewline
\hline 
$X_{O}=\left(\begin{array}{ccccc}
1 & \varphi_{1}(x_{1}) & \varphi_{2}(x_{1}) & \cdots & \varphi_{t}(x_{1})\\
\vdots & \vdots & \vdots & \ddots & \vdots\\
1 & \varphi_{1}(x_{n}) & \varphi_{2}(x_{n}) & \cdots & \varphi_{t}(x_{n})\end{array}\right)$\tabularnewline
\hline 
$X'X$ is a diagonal matrix, for computation convenience\tabularnewline
\hline 
$M(\beta)=RSS(\beta)$\tabularnewline
\hline
\end{tabular}

\begin{tabular}{|c|}
\hline 
Piecewise Polynomial Regression\tabularnewline
\hline 
\tabincell{c}{
partition the domain of $t$ into consequential intervals(add knots),\\
in each interval use a polynomial to approximate $f$($()_{+}$)}\tabularnewline
\hline 
$M=(X_{P}Z)$;$Z=\left(\begin{array}{ccc}
(x_{1}-\xi_{1})_{+}^{m-1} & \cdots & (x_{1}-\xi_{k})_{+}^{m-1}\\
\vdots & \ddots & \vdots\\
(x_{n}-\xi_{1})_{+}^{m-1} & \cdots & (x_{n}-\xi_{k})_{+}^{m-1}\end{array}\right)$;$x_{+}^{r}=\{max(0,x)\}^{r}$\tabularnewline
\hline 
need to select knots(number\&location)\tabularnewline
\hline 
$M(\beta)=RSS(\beta)$\tabularnewline
\hline
\end{tabular}
\newpage
\begin{tabular}{|c|}
\hline 
Smoothing Spline\tabularnewline
\hline
$Y=X\beta_{P}+Zu+\epsilon=\beta B(x)+\epsilon$\tabularnewline
\hline 
$B(x)=[B_{1}(x),\cdots,B_{N}(x)]^{T}$, vector of spline basis functions\tabularnewline
\hline 
knots are data points, need to select $\alpha$\tabularnewline
\hline 
$M(\beta)=\Sigma_{i=1}^{n}\{y_{i}-\beta^{T}B(x_{i})\}^{2}+\alpha\beta^{T}D\beta$;$D=\int_{a}^{b}B^{(q+1)}(x)\{B^{(q+1)}(x)\}^{T}dx$\tabularnewline
\hline
\end{tabular}	

\begin{tabular}{|c|}
\hline 
Penalized Spline		\tabularnewline
\hline
$Y=X\beta_{P}+Zu+\epsilon=\beta B(x)+\epsilon$\tabularnewline
\hline
\tabularnewline
\hline 
need to select knots and $\alpha$\tabularnewline
\hline 
$M(\beta)=\Sigma_{i=1}^{n}\{y_{i}-\beta^{T}B(x_{i})\}^{2}+\alpha\beta^{T}D\beta$\tabularnewline
\hline
\end{tabular}	

\section{Regression Spline/Least-square Spline}

$u(t)\sim
S(t)=\Sigma_{i=1}^{m}\theta_{j}t^{j-1}+\Sigma_{j=1}^{k}\delta_{j}(t-\xi_{j})_{+}^{m-1}$

order: $m$

knots: $\xi_{1},\cdots,\xi_{k}$: $\xi=\{\xi_{1},\cdots,\xi_{k}\}$

coefficients: $\delta_{1},\cdots\delta_{k}$

\subsection{least square spline estimator of $u$}

$u=\Sigma_{j=1}^{m+k}b_{\xi j}x_{j}$, where

$x_{j}(t)=t^{j-1}$, $j=1,\cdots,m$

$x_{m+j}(t)=(t-\xi_{j})_{+}^{m-1}$, $j=1,\cdots,k$$b_{\xi}$: the estimator of 

$\beta=(\theta_{1},\cdots,\theta_{m},\delta_{1},\cdots,\delta_{k})^{T}$,

obtained by minimizing
$RSS(c;\xi)=\Sigma_{i=1}^{n}(y_{i}-\Sigma_{j=1}^{m+k}c_{j}x_{j}(t))^{2}$,

with respect to $c=(c_{1},\cdots,c_{m+k})^{T}$

$X_{\xi}^{T}X_{\xi}c=X_{\xi}^{T}Y$,$X_{\xi}=\{x_{j}(t_{i})\}i=1,\cdots,n_{j};j=1,\cdots,m+k$;

$b_{\xi}=(X_{\xi}^{T}X_{\xi})^{-1}X_{\xi}Y$

\subsection{selecting $\xi$:} 
number and location of the knots for the estimator(m=2,3,(4))

1. visual inspection of the data: more knots will be needed where
$u$ seems to change more rapidly

2. variable projection method: minimizing $RSS(\xi)=\Sigma_{i=1}^{n}(y_{i}-\Sigma_{j=1}^{m+k}b_{\xi j}x_{j}(t_{j}))^{2}$,
$\xi=\{\xi_{1},\cdots,\xi_{k}\}$arbitrary choice for knots set

3. GCV: data driven choice for both location and number of knots

$\xi(K)=\{\xi_{1},\cdots,\xi_{K}\}$, $GCV[\xi(K)]=\frac{n^{-1}RSS[\xi(K)]}{[1-(m+K)/n]^{2}}$

fixed $K$ ->$\hat{\xi}(K)$ ->$\hat{\xi}(\hat{K})$ -> ...

4. $MDL$

\subsection{Do not admit kernel or series representations}

\subsection{Large sample properties:}

{[}Shanggang Zhou, Douglas A. Wolfe, On derivative estimation in spline
regression{]}

\section{1.2 Smoothing Spline}

$y_{i}=u(x_{i})+\epsilon_{i}$

$M(\lambda)=\Sigma[y_{i}-\hat{u_{n}}(x_{i})]^{2}+\lambda J(u)$

$J(u)$: roughness penalty(regulation), $=\int[u''(x)]^{2}dx$

\subsection{Definition:}
Knots: $\xi_{1}<\xi_{2}<\cdot\cdot\cdot<\xi_{k}$ be a set of ordered
points contained in some interval $(a,b)$.

$M$th-order spline: a piecewise$M\lyxmathsym{\textminus}1$ degree(=order-1)
polynomial with $M\lyxmathsym{\textminus}2$ continuous derivatives
at the knots. 

Cubic spline($M=4$): a continuous function $u$ such that (i) $u$
is a cubic polynomial over $(\xi_{1},\xi_{2}),...$ and (ii) $u$
has continuous first and second derivatives at the knots.

Natural Spline: a spline that is linear beyond the boundary knots.

\subsection{Theorem:}

The function $u_{n}(x)$ that minimizes $M(\lyxmathsym{\textgreek{l}})$
with penalty $\int[u''(x)]^{2}dx$ is a natural cubic spline with
knots at the data points(so there is no need to select the knots,
all knots are data points, number of knots are the number of observations,
this is the difference from regression splines). The estimator $u_{n}$
is called a smoothing spline.

\subsection{Basis:} 

To give an explicit form for $u_{n}$we need to construct a basis
for the set of splines.

{[}Proofs that they are basis: de Boor, A Practical Guide to Splines
(revised edition) (2001){]}

Truncated power basis: (convenience for representation)

A basis for the set of cubic splines at knots $\xi$: ${h_{1},...,h_{k+4}}:h_{1}(x)=1,h_{2}(x)=x,h_{3}(x)=x^{2},h_{4}(x)=x^{3},h_{j}(x)=(x\lyxmathsym{\textminus}\xi_{j}\lyxmathsym{\textminus}4)_{+}^{3}$
for $j=5,...,k+4$. 

$u(x)=\Sigma_{j=1}^{k+4}\beta_{j}h_{j}(x).$

B-spline basis: (convenience for computation)

\subsection{Estimation:}

$\hat{u_{n}}(x)=\Sigma_{i=1}^{N}\hat{\beta_{j}}B_{j}(x)$, $B_{1},\cdots,B_{N}$:
natural spline basis

$M(\beta)=(Y-B\beta)^{T}(Y-B\beta)+\lambda\beta^{T}\Omega\beta$, 

$\beta=(\hat{\beta_{1}},\cdots,\hat{\beta_{N}})^{T}$

$B_{ij}=B_{j}(x_{i})$

$\Omega_{jk}=\int B_{j}^{''}(x)B_{k}^{''}(x)dx$

we have 

$\hat{\beta}=(B^{T}B+\lambda\Omega)^{-1}B^{T}Y$

\subsection{Selection of $\lambda$}

{[}Randall L. Eubank, Nonparametric Regression and Spline Smoothing,
P239{]}

1. minimization of unbiased risk criterion $\hat{P}(\lambda)=\frac{1}{n}RSS(u_{\lambda})+\frac{2\hat{\sigma}}{n}trS_{\lambda}$,
$S_{\lambda}=X(B^{T}B+\lambda\Omega)^{-1}B^{T}$, $\hat{\sigma}:$GSJS
estimator

2. $CV=n^{-1}\Sigma_{i=1}^{n}[y_{i}-u_{\lambda(i)}t(i)]^{2}$, $u_{\lambda(i)}$:
estimator when $i$th observation $(t_{i},y_{i})$ deleted from data

3. $GCV=\frac{nRSS(u_{\lambda})}{tr(I-S_{\lambda})}$

4. $MDL$

\subsection{Other properties}Admit kernel/series representations, 

can be interpreted as minimum variance unbiased linear prediction
pf a class of stochastic process{[}George S. Kimeldorf, Grace Wahba,
Spline functions and stochastic processes, 1970{]}. 

\subsection{ Large sample properties:}

{[}Randall L. Eubank, Nonparametric Regression and Spline Smoothing,
P247{]}

$n\to\infty$, $\lambda\to0$; $\lambda$is similar to the bandwidth
$h$ in kernel estimation

use a global, kernel type approximation for linear smoothing spline

\subsection{Interpretation as splines with penalty:}

{[}Randall L. Eubank, Nonparametric Regression and Spline Smoothing,
P227{]}

\subsection{Interpretation using Taylor's theorem:}

{[}Randall L. Eubank, Nonparametric Regression and Spline Smoothing,
P228{]}

\section{1.3 Penalized Spline}

{[}Eilers, Marx, Flexible smoothing with B-splines and penalties{]}:
relatively large number of knots and a difference penalty on coefficients
of adjacent B-splines.

differences from Smoothing Spline: (i) less knots (ii) general penalties

\subsection{General Definition of a Penalized Spline:}

is $\hat{\beta}^{T}B(x)$, where $\hat{\beta}$is the minimizer of
$\Sigma_{i=1}^{n}\{y_{i}-\beta^{T}B(x_{i})\}^{2}+\alpha\beta^{T}D\beta$,
for some symmetric positive semidefinite matrix $D$ and scalar $\alpha>0$.

\subsection{Steps for applying penalized splines:}


(1) Spline Model: degree, knot locations, constrains (such as natural
spline boundary constrains);

(2) Penalty: form of penalty up to a nonnegtive smoothing parameter

(3) Basis functions: (truncated power functions or B-splines) used
to represent the model, used in the computations; (do not affect the
fitted curve)

(2+3) Penalty matrix $D$: automatically determined once the Penalty(form)
and Basis functions have been determined.

\subsection{Penalized splines in use:}


1. Large sample properties:

{[}Gerda Claeskens, Tatyana Krivobokova, Jean D. Opsomer, Asymptotic
properties of penalized spline estimators{]}

or

{[}Hulin Wu, Hongqi Xue, Arun Kumar, Numerical discretization-based
estimation methods for ordinary differential equation models via penalized
spline smoothing{]}

2. Working assumption: nonparameter estimation->parameter estimation

Reason: {[}Yan Yu, David Ruppert, Penalized Spline Estimation for
Partially Linear Single-Index Models{]}

{[}Yang Bai, Wing K. Fung, Zhong Yi Zhu, Penalized quadratic inference
functions for single-index models with longitudinal data{]}

$\eta_{0}$:$(R\to R)$ unknown univariate function

suppose $\eta_{0}$ is a $p$th degree spline function with knots
$\xi_{1},\cdots,\xi_{K}$, spline basis $B(x)=(1,x,\cdots,x^{p},(x-\xi_{1})_{+}^{p},\cdots,(x-\xi_{K})_{+}^{p})^{T}$

$\hat{\eta}_{0}\approx B\delta_{0}$, is the best projection of the
true smoothing function$\eta_{0}$on the space$B\delta$. $\delta_{0}$
are spline coefficients, spline parameter. We call $\delta_{0}$ the
{}``true'' parameter.

If there are other parameters($\beta$) to estimate, we represent
all the parameters as $\theta=(\beta^{T}\delta_{0}^{T})^{T}$

\subsection{selection of knots and $\lambda$}


1. {[}David Ruppert, Selecting the Number of Knots For Penalized Splines{]}

{[}David Ruppert, M. P. Wand, R. J. Carroll, Semiparametric Regression{]}

2. $MDL$

\subsection{Penalized spline and shrinkage:}

{[}David Ruppert, M. P. Wand, R. J. Carroll, Semiparametric Regression,
P74{]}

\section{References:}

\subsection{Books:}


Semiparametric Regression by David Ruppert, M. P. Wand, R. J. Carroll,
2003

Nonparametric Regression and Spline Smoothing, Second Edition, by
Randall L. Eubank, 1999

All of Nonparametric Statistics, by Larry Wasserman, 2005

A Practical Guide to Splines (revised edition), by Carl de Boor, 2001

The Elements of Statistical Learning: Data Mining, Inference, and
Prediction. Second Edition, by Trevor Hastie, Robert Tibshirani, Jerome
Friedman, 2009

\subsection{Papers:}


Flexible smoothing with B-splines and penalties, Eilers, Marx, Statistical
Science, 1996

Spline functions and stochastic processes, George S. Kimeldorf, Grace
Wahba, The Indian Journal of Statistics, 1970

On derivative estimation in spline regression, Shanggang Zhou, Douglas
A. Wolfe, Statistica Sinica, 2000

Asymptotic properties of penalized spline estimators, Gerda Claeskens,
Tatyana Krivobokova, Jean D. Opsomer, Biometrika, 2009,

Numerical discretization-based estimation methods for ordinary differential
equation models via penalized spline smoothing, Hulin Wu, Hongqi Xue,
Arun Kumar, submitted to the Annals of Statistics 

Penalized Spline Estimation for Partially Linear Single-Index Models,
Yan Yu, David Ruppert, Journal of the American Statistical Association,
2002 

Penalized quadratic inference functions for single-index models with
longitudinal data, Yang Bai, Wing K. Fung, Zhong Yi Zhu, Multivariate
Analysis, 2009

Selecting the Number of Knots For Penalized Splines, David Ruppert,
2000
\end{document}
