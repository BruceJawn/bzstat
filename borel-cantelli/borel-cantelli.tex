% Ԥ��Դ�ļ�

%% *March 31, 2011
%% *Bruce Zhou
%% *http://bzstat.wordpress.com

%% *Copyright (c) <2011> <Bruce Zhou>
%% *This software is released under the MIT License
%% *<http://www.opensource.org/licenses/mit-license.php>
%%
\documentclass[english]{article}
\usepackage{lmodern}
\usepackage[T1]{fontenc}
\usepackage[latin9]{inputenc}
\usepackage{textcomp}
\usepackage{amstext}
\usepackage{amssymb}
\usepackage{CJK}
%\makeatletter

%%%%%%%%%%%%%%%%%%%%%%%%%%%%%% LyX specific LaTeX commands.
\providecommand*{\perispomeni}{\char126}
\AtBeginDocument{\DeclareRobustCommand{\greektext}{%
  \fontencoding{LGR}\selectfont\def\encodingdefault{LGR}%
  \renewcommand{\~}{\perispomeni}%
}} \DeclareRobustCommand{\textgreek}[1]{\leavevmode{\greektext #1}}
\DeclareFontEncoding{LGR}{}{}

\DeclareRobustCommand{\cyrtext}{%
  \fontencoding{T2A}\selectfont\def\encodingdefault{T2A}}
\DeclareRobustCommand{\textcyr}[1]{\leavevmode{\cyrtext #1}}
\AtBeginDocument{\DeclareFontEncoding{T2A}{}{}}

\newcommand{\lyxmathsym}[1]{\ifmmode\begingroup\def\b@ld{bold}
  \text{\ifx\math@version\b@ld\bfseries\fi#1}\endgroup\else#1\fi}


%\makeatother

\usepackage{babel}

\begin{document}
\begin{CJK}{GBK}{song}
\title{On Borel\textendash{}Cantelli Lemma}
%\begin{center}
%\author{}
%\end{center}
\date{}
\maketitle

Borel\textendash{}Cantelli lemma is a theorem about sequences of
events. It is named after emile Borel and Francesco Paolo Cantelli
([E. Borel,1909,F.P. Cantelli,1917]). The Borel\textendash{}Cantelli
lemma states that:
\newline(1) If
$\Sigma_{n=1}^{\infty}P(A_{n})<\infty$, then
$$P\{A_{n}i.o.\}=P(\limsup A_{n})=0.$$
(2) If $A_{n}$ are independent and
$\Sigma_{n=1}^{\infty}P(A_{n})=\infty$, then
$$P\{A_{n}i.o.\}=P(\limsup A_{n})=1.$$

In this paper, we review some extensions of the original
Borel\textendash{}Cantelli Lemma.

In Barndorff-Nielsen's paper ([Barndorff-Nielsen,1961]), a general
form of the Law of the Iterated Logarithm for Maximal Order
Statistics is established by means of a generalization of the
convergence part of the Borel-Cantelli lemmas. The generalized Lemma
is as follows:

\textbf{Lemma:} If $$\liminf_{n\to\infty}P(A_{n})=0$$ and
$$\sum_{n=1}^{\infty}P(A_{n}\cap A_{n+1}^{c})<\infty,$$ then one has
$$P(\limsup A_{n})=0.$$

Proof: Set $B_{n}=A_{n}\cap A_{n+1}^{c}$, $E=\limsup A_{n}$ and
$F=\limsup A_{n}^{c}$. We have $$P(\limsup B_{n})\leq
P(F^{c})=P(\liminf A_{n})\leq\liminf_{n\to\infty}P(A_{n})=0.$$

Observe that $E\cap F\subset\limsup B_{n}$. To see this, fix
$n\geq1$ and $\omega\in E$, $\exists m\geq n$ such that $\omega\in
A_{m}$. Put $l=inf\{k\geq m:\omega\in A_{k}^{c}\};$ by definition of
$F$,

$l<\infty$ and $\omega\in A_{l}\in A_{l-1}^{c}$. Since $n$ is
arbitrary, $\omega\in\limsup B_{n}$. Hence $$P(E)\leq
P(F^{c})+P(E\cap F)=0.$$

The Extended Renyi-Lamperti lemma relaxes the assumption of
independence.

\textbf{Theorem:} If $$\sum_{n=1}^{\infty}P(A_{n})=\infty$$ and
$$\liminf_{n\to\infty}\frac{\sum_{j=1}^{n}\sum_{k=}^{n}P(A_{i}\cap
A_{j})}{(\sum_{j=1}^{n}P(A_{j}))^{2}}=c,$$ then $$P(\limsup
A_{n})\geq1/c.$$

Proof: Recall that if $EX>0$ and $0\leq\epsilon<1$, applying
Cauchy-Schwarz inequality to $E(XI\{X>\epsilon EX\})$ we have
$$P(X>\epsilon EX)\geq(1-\epsilon)^{2}(EX)^{2}/EX^{2}.$$

Put $J_{n}=\Sigma_{j=1}^{n}I(A_{j})$ and
$s_{n}=\Sigma_{j=1}^{n}P(A_{j})=EJ_{n}$. Fix $0<\epsilon<1$ and put
$B_{n}=\{J_{n}\geq\epsilon s_{n}\}$. As $s_{n}\to\infty$, $\limsup
B_{n}\subset\limsup A_{n}$. Now
$P(B_{n})\geq(1-\epsilon)^{2}(EJ_{n})^{2}/EJ_{n}^{2}$, and so
$\limsup_{n\to\infty}P(B_{n})\geq(1-\epsilon)^{2}/c$. As
$\epsilon>0$ is arbitrary, the proof is complete.

T Chandra and S Ghosal ([T. K. Chandra,1993]) gave another extension
in 1993.

\textbf{Theorem:} Suppose $\{A_{n}\}$ is a sequence of events
satisfying $$P(A_{i}\cap A_{j})-P(A_{i})P(A_{j})\leq
q(j-i)P(A_{j}),i<j,$$ where $q(m)\geq0,\forall m\geq1$ and
$$\frac{\sum_{m=1}^{\infty}q(m)}{\sum_{j=1}^{m}P(A_{j})}\leq\infty.$$
If $$\sum_{n=1}^{\infty}P(A_{n})=\infty,$$ then $$P(\limsup
A_{n})=1.$$

A Borel-Cantelli lemma for {*}-mixing sequences was given by J. R.
BLC\textasciitilde{}, D. L. HANSON and L. II. KOOPMANS ([J. R.
BLC,1963]).

\textbf{Definition:} The sequence $\{X_{n}\}$ will be called
{*}-mixing if there exists a positive integer $N$ and a real-valued
function $f$ defined for the integers $n\geq N$ such that (i) $f$ is
non-increasing with $lim_{n\to\infty}f(n)=0$, and (ii) if $n\geq N$,
$$A\in\sum_{1}^{m}(\sigma<x_{1},\cdots x_{m}>),$$
$$B\in\sum_{m+n}(\sigma<x_{m+n}>),$$ then $$|P(AB)-P(A)P(B)|\leq
f(n)P(A)P(B).$$

\textbf{Theorem:} Let $\{A_{n}\}$ be a sequence of events such that
$\{I(A_{n})\}$ is {*}-mixing and
$$\sum_{n=1}^{\infty}P(A_{n})=\infty.$$ Then $$P(\limsup A_{n})=1.$$

Proof: Let $0<\delta<1$ and get $k>N$ such that $f_{k}<\delta$. Get
$1\leq j\leq k$ such that $\Sigma_{n=0}^{\infty}P(A_{nk+j})=\infty$.
Set $B_{n}=A_{nk+j}$. It suffices to show that $P(\limsup B_{n})=1$.
If not, there exists $m\geq1$ such that $P(\cup_{i=1}^{m}B_{i})<1$.
Hence

$$P(\cap_{i=1}^{m}B_{i})=P(B_{m})+\Sigma_{i=1}^{\infty}P(B_{m+t}\cap_{s=0}^{t-1}B_{m+s}^{c})$$

$$\geq(1-\delta)[P(B_{m})+\Sigma_{t=1}^{\infty}P(B_{m+t})P(\cap_{s=0}^{t-1}B_{m+s}^{c})]$$

$$\geq(1-\delta)P(\cap_{s=0}^{t-1}B_{m+s}^{c})\Sigma_{t=0}^{\infty}P(B_{m+t})=\infty,$$
which is a contradiction.

The following version of Borel-Cantelli lemma is due to R. J.
Serfling([R. J. Serfling,1975]), which obtained a new extension for
dependent events of the divergent part of the Borel-Cantelli lemma.

\textbf{Theorem:} Let $\{A_{n}\}$ be a sequence of events and let
$$\mathit{\mathcal{F}}_{n}=\sigma<A_{1},\cdots,A_{n}>.$$Then
$$P\{(\limsup
A_{n}\Delta\{\sum_{n=1}^{\infty}P(A_{n}|\mathcal{F}_{n-1})=\infty\}\}=0,$$
where $\Delta$ stands for the symmetric difference.

Proof: [R. J. Serfling,1975] has shown that for all $m<N$,
$$P(\cap_{n=m}^{N}A_{n}^{c})\leq
exp[-\Sigma_{n=m}^{N}P(A_{n})]+\Sigma_{n=m}^{N}E|P(A_{n}|\mathcal{\mathsf{\mathfrak{\mathrm{F}}}}_{n-1})-P(A_{n})|.$$

Let $N\to\infty$ and then $m\to\infty$, the results follows.


%///
In 1970, J. Shuster([Shuster, 1970]) provided a generalized version
of the first and the second Borel-Cantelli Lemma through giving
lower bound for $p(\cap_{n=1}^{\infty}\cup_{k=n}^{\infty}A_k)$ The
theorem below is the main result of J.Shuster's article.\newline
\textbf{Theorem:}\newline (a) If there exists an $A\in\textit F$
such that
$$\sum_{k=1}^{\infty}P(A\cap A_k)<\infty$$
then $$P(A_n i.o.)\leq 1-P(A)$$ (b) If for every set $A\in\textit F$
such that $P(A)>0$ it holds that
$$\sum_{k=1}^{\infty}P(A\cap A_k)=\infty$$
then $$P(A_n i.o.)=1$$ proof of (a)  To see that (a) theorem holds,
we can look at the indicator function $I_k$ of the event $A\cap
A_k$. Now, define $T=\sum_{k=1}^{\infty}I_k $. Consider the expected
value of T. Because $E(T)=\sum_{k=1}^{\infty}P(A\cap A_k)$ and, by
the hypothesis of (a),$\sum_{k=1}^{\infty}P(A\cap A_k)=\infty$ , we
have $E(T)< \infty$. But this means that only a finite number of
$A\cap A_k$occur. This means that the probability $P(A_n i.o.)$is at
most equal to $P(A^c )=1-P(A)$, which is what we wanted to prove.
Proof of (b) To see why (b) is true, define the set
$$B_n=\cap_{k=n}^{\infty}A_{k}^{c}=A_{n}^{c}\cap A_{n+1}^{c}\cap
A_{n+2}^{c}\cap \ldots$$ If we look at the event $A_k\cap B_n$, we
can see that if $k\ge n$then $P(A_k\cap B_n)=0$ This means that
$$\sum_{k=1}^{\infty}P(A_k\cap B_n)=\sum_{k=1}^{n-1}P(A_k\cap B_n)$$
Note that the sum on the right-hand side is finite. This implies
that we must have that $P(B_n)=0$ and this means that
$$P(\cup_{n=1}^{\infty}B_n)=P(\cup_{n=1}^{\infty}A_{k}^{c}=0)\rightrightarrows
P(\cap_{n=1}^{\infty}\cup_{k=n}^{\infty}A_k)=1$$ and this is what we
wanted to show.\newline

Clearly, this theorem is a generalization of the original
Borel-Cantelli Lemmas. When we set $A=\Omega$, (a) is precisely what
the original first Borel-Cantelli Lemma stated. Meanwhile, if we
compare (b)  with the original second Borel-Cantelli Lemma, we can
see that the independence criterion is dropped altogether.
%///

A generalization of the Erdos\textendash{}Renyi formulation of the
Borel\textendash{}Cantelli lemma is obtained by Valentin V.Petrov
([Valentin V.Petrov,2002] and
\newline[Valentin V.Petrov,2004]).

\textbf{Theorem A:} Let $A_{1},A_{2}\cdots$ be a sequence of events
satisfying conditions

$\sum_{n=1}^{\infty}P(A_{n})=\infty$ and $P(A_{k}A_{j})\leq
CP(A_{k})P(A_{j})$ for all $k,j>L$ such that $k=j$ and for some
constants $C\geq1$ and $L$. Then $P(\limsup A_{n})\geq1/C.$

\textbf{Theorem B:} Let $A_{1},A_{2}\cdots$ be a sequence of events
satisfying conditions

$$\sum_{n=1}^{\infty}P(A_{n})=\infty.$$ Let $H$ be an arbitrary real
constant. Put

$$\alpha_{H}=\liminf\frac{\sum_{1\leq i<k\leq
n}(P(A_{i}A_{k})-HP(A_{i})P(A_{k}))}{(\sum_{k=1}^{n}P(A_{k}))^{2}}.$$
Then $$P(\limsup A_{n})\leq\frac{1}{H+2\alpha_{H}}.$$
%///
\textbf{Remark:} $P(A_{n}i.o.)=\alpha$ and equivalent statements

\textbf{Theorem:} Let $0<\alpha\leq1$. The following statements are
equivalent:

1. $P(A_{n}i.o.)\geq\alpha$

2. $\Sigma_{n=1}^{\infty}P(A_{n}\cap B)=\infty$ for any
$B\in\mathcal{F}$ with $P(B)>1-\alpha$

3. for any $B\in\mathcal{F}$ with $P(B)>1-\alpha$, the sequence
$\{P(A_{n}\cap B)\}$ contains an infinite number of positive
numbers.

Proof: First note that (2)$\Rightarrow$(3),

if not, only a finite number of $P(A_{n}\cap B)>0$ so
$\Sigma_{n=1}^{\infty}P(A_{n}\cap B_{n})\neq\infty$.

(1)$\Rightarrow$(2):

Suppose $P(A_{n}i.o.)\geq\alpha$ but assume that
$\Sigma_{n=1}^{\infty}P(A_{n}\cap B)<\infty$. Then according to
Borel-Cantelli lemma. $P(A_{n}i.o.)+P(B)-1\leq P(A_{n}\cap
Bi.o.)=0$. This implies $P(B)\leq1-\alpha$.

(3)$\Rightarrow$(1): Suppose that (3) holds but
$P(A_{n}i.o.)<\alpha$, we have $P(\cup_{k=n}^{\infty}A_{n})<\alpha$
for some $n$. Define $C=\cup_{k=n}^{\infty}A_{k}$ and $B=C^{c}$,
then $P(B)>1-\alpha$ and $B\cap
C=B\cap\cup_{k=n}^{\infty}A_{k}=\phi$, which means $P(A_{k}\cap
B)=0$ for all $k\geq n$. This is a contradiction.
%///
 WenLiu, Jia-anYan
and Weiguo Yangc generalized the extended Borel\textendash{}Cantelli
lemma as corollaries([WenLiu,2003]).

\textbf{Corollary:} Let $(\mathcal{F}_{n},n\geq0)$ be an increasing
sequence of $\sigma$-algebras and let$B_{n}\in\mathcal{F}_{n}$. Put
$$A=\{\sum_{i=1}^{\infty}I_{B_{i}}=\infty\};B=\{\sum_{i=1}^{\infty}P(B_{i}|\mathcal{F}_{i-1})=\infty\}.$$
Then we have $A=B$ a.s., and
$$\lim_{n\text{\textrightarrow\ensuremath{\infty}}}\frac{\sum_{i=1}^{n}I_{Bi}}{\sum_{i=1}^{n}P(B_{i}|\mathcal{F}_{i-1})}=1$$
a.s. on B.

The first part of this corollary is the Extended
Borel\textendash{}Cantelli Lemma
\\
({Chow,1988}), and the second part of this corollary is the sharper
form of the Borel\textendash{}Cantelli lemma ({Dubins,1965}).

Generalizations of the second Borel\textendash{}Cantelli lemma are
obtained under very weak dependence conditions by Tapas Kumar
Chandra({Tapas Kumar Chandra,2008}), subsuming several earlier
results as special cases.

Let $\{A_{n}\}n\geq1$, satisfy

$$\Sigma_{n=1}^{\infty}P(A_{n})=0$$ and
$$\liminf_{n\to\infty}\frac{\Sigma_{1\leq j<k\leq n}(P(A_{j}\cap
A_{k})-a_{ij})}{(\Sigma_{1\leq k\leq n}P(A_{k}))^{2}}=L,$$

where $a_{ij}$ satisfy $$\Sigma_{1\leq j<m\leq k\leq
n}|a_{ij}|=o((\Sigma_{m\leq k\leq n}P(A_{k}))^{2})$$ $\forall
m\geq1$ and
$$\liminf_{m\to\infty}\limsup_{n\to\infty}\frac{\Sigma_{1\leq j<k\leq
n}a_{jk}}{(\Sigma_{m\leq k\leq n}P(A_{k}))^{2}}\leq d.$$ Assume that
$L$ and $d$ are finite. Then $$d+L\geq\frac{1}{2}$$ and $$P(\limsup
A_{n})\geq(2d+2L)^{-1}.$$

Borel-Cantelli lemma when the condition of independent events is
replaced by negative quadrant dependent condition was discussed by
Gholam Hossein Yari and Farhad Hossein Zadeh([Gholam Hossein
Yari,2010]).

\textbf{Definition:} A sequence of random variables ${X_{n};n\ge1}$
is said to be pairwise negative quadrant dependent (NQD) if,
$$P(X_{i}\leq x,Y_{j}\leq y)\leq P(X_{i}\leq x)P(Y_{j}\leq y)$$ for
all $x,y\in R$ and for all $i,j\ge1$, $i\text{\ensuremath{\neq}}j$.

\textbf{Theorem:} Let ${A_{n}}_{n=1}^{\infty}$ be a sequence of
events in a probability space
$$(\lyxmathsym{\textgreek{W}},\lyxmathsym{\textgreek{A}},\lyxmathsym{\textgreek{R}})$$
and set
$$\limsup_{n\text{\textrightarrow\ensuremath{\infty}}}A_{n}=A.$$
If
$$\lyxmathsym{\textgreek{S}}_{n=1}^{\infty}P(A_{n})=\text{\ensuremath{\infty}},$$
and
$$\limsup_{n\to\infty}\frac{\text{(\textgreek{S}}_{i=1}^{\text{\ensuremath{\infty}}}P(A_{i}))^{2}}{\Sigma_{i=1}^{n}\text{\textgreek{S}}_{j=1}^{n}P[A_{i}\cap
A_{j}]}=\alpha>0,$$ then, (a)$P[A]\ge\lyxmathsym{\textgreek{a}}$.
(b) If the events are pairwise (NQD) then $P[A]=1$.

From those theorems above, we can see extended
Borel\textendash{}Cantelli lemmas can relax parts of the assumptions
in the original one such as independence. However, the drawback is
obvious that we either need some more complicated conditions or can
only get some weaker results. The beauty in the original
Borel\textendash{}Cantelli lemma's simplicity and interpretability
is lost. Many of those theorem are derived as lemmas for varied
purposes or specified problems. Although it seems that most of those
lemmas are more general, they're not as popular as the original one
because those extended lemmas are more difficult to explain and
remember.
\newline
\newline
\begin{center}
\textbf{References} \end{center}
[1][E. Borel, Denumerable
probabilities and their applications arithmetic(in French),
\newline Rend. Circ. Mat. Palermo (2) , 27 (1909) pp. 247-271]\{E. Borel,
1909-1-1\}
\newline
[2][F.P. Cantelli, On probability as a frequency limit(in Italian),
\newline Atti Accad. Naz. Lincei , 26 : 1 (1917) pp. 39-45.]\{F.P.
Cantelli, 1917-1-1\}
\newline
[3][Subhashis Ghoshal, T. K. Chandra, On Borel-Cantelli lemmas.
\newline In Essays on Probability and Statistics, Festschrift in honour of
Professor Anil Kumar  Bhattacharya (S. P. Mukherjee et al. (eds.)),
Department of Statistics, Presidency College, Calcutta, pp. 231-239,
1994.]\{Subhashis Ghoshal, 1994\}
\newline
[4][S. de Vos, On Generalizations Of The Borel-Cantelli Lemmas,
\newline bachelor thesis of university of groningen, 2010]\{Vos, 2010\}
\newline
[5][Barndorff-Nielsen, On the rate of growth of the partial maxima
of a sequence of independent identically distributed random
variables,
\newline Mathematica Scandinavica, 1961.]\{Barndorff-Nielsen, 1961\}
\newline
[6][T. K. Chandra, S. Ghosal, Some elementary strong laws of large
numbers: a review,
\newline Tech. Report, Indian Statistical Institute (1993).]\{T. K. Chandra,
1993\}
\newline
[7][J. R. BLC~, D. L. HANSON, L. II. KOOPMANS On the Strong Law of
Large Numbers for a Class of Stochastic Processes,
\newline Z. Wahrscheinlichkeitstheorie 2, 1-11 (1963).]\{J. R. BLC, 1963\}
\newline
[8][R. J. Serfling, A General Poisson Approximation Theorem,
\newline The Annals of Probability, Vol. 3, No. 4 (Aug., 1975), pp.
726-731]\{R. J. Serfling, 1975\}
\newline[9][J. Shuster, On the Borel-Cantelli problem,
\newline Canadian Mathematical Bulletin 13 (1970), pages 273-275]\{Shuster, 1970\}
\newline[10][Valentin V.Petrov, A note on the Borel-Cantelli lemma,
\newline Statistics \& Probability Letters 58 (2002) 283-286]\{Valentin
V.Petrov, 2002\}
\newline
[11][Valentin V.Petrov, A generalization of the Borel-Cantelli
Lemma,
\newline Statistics \& Probability Letters 67 (2004) 233-239]\{Valentin
V.Petrov, 2004\}
\newline
[12][WenLiu, Jia-anYan, Weiguo Yangc, A limit theorem for partial
sums of random variables and its applications,
\newline Statistics \& Probability Letters 62 (2003) 79-86]\{WenLiu, 2003\}
\newline
[13][Chow, Teicher, Probability Theory, 2nd Edition.
\newline Springer, New York, (1988) p. 249]\{Chow, 1988\}
\newline
[14][Dubins, Freedman, A sharper form of the Borel-Cantelli lemma
and the strong law.
\newline Ann. Math. Statist 36, (1965) 800-807]\{Dubins, 1965\}
\newline
[15][Tapas Kumar Chandra, The Borel-Cantelli lemma under dependence
conditions,
\newline Statistics \& Probability Letters 78 (2008) 390-395]\{Tapas Kumar
Chandra, 2008\}
\newline
[16][Gholam Hossein Yari, Farhad Hossein Zadeh, Extend the
Borel-Cantelli Lemma to Sequences of Non-Independent Random
Variables,
\newline Applied Mathematical Sciences, Vol. 4, 2010, no. 13, 637-642]
\{Gholam Hossein Yari, 2010\}
\end{CJK}
\end{document}
